\documentclass[12pt]{article}
\usepackage{amsmath,amsfonts,amsthm}
\usepackage[fleqn]{mathtools}
\usepackage{a4wide}
\usepackage{fancyhdr} % Custom headers and footers
\usepackage{graphicx}
\usepackage{listings}
\usepackage{parskip}

\usepackage[adobe-utopia]{mathdesign}
\usepackage[T1]{fontenc}

\usepackage{titlesec} % www.ctan.org/tex-archive/macros/latex/contrib/titlesec/titlesec.pdf

% Below "\section" can be replaced with "\subsection" and "\subsubsection"
% in order to customize the corresponding headings



\pagestyle{fancyplain}
\fancyhead[L]{HW1: 03/03/2014}
\fancyhead[C]{University of Wisconsin-Madison}
\fancyhead[R]{Professor Eric Bach}
\fancyfoot[L]{Taylor Lee}
\fancyfoot[C]{CS 812}
\fancyfoot[R]{tdlee2@wisc.edu}
\newtheoremstyle{moo}
{10pt}
{10pt}
{}
{}
{\bfseries}
{:}
{\newline }
{}


\theoremstyle{moo}
\newtheorem*{prob}{Problem}
\newtheorem*{sol}{Solution}
\newtheorem*{theorem}{Theorem}

\def\zz{{\mathbb Z}}
\def\lr{{\Leftrightarrow}}
\def\poi{{\pmod{26}}}


\begin{document}
\fontencoding {T1}
\fontfamily {put}
\fontseries {\seriesdefault}
\fontshape {\shapedefault}
%\fontsize {size} {baselineskip} 
\selectfont

\title{\usefont{T1}{put}{b}{n} Arithmetic Algorithms 2014: HW2}
\date{March 3, 2014}         % used by \maketitle
\author{Taylor Lee}      % used by \maketitle                    % used by \maketitle
\maketitle                      % automatic title!



























\subsection*{Part B:}

Here, since $a$ again has an inverse $\pmod{p}$, and hence higher powers of $p$, we can start our lift again with Newton Iteration. Thus, $a - \frac{1}{x^2} = 0 \pmod{p^k}$ gives us:

\[
x_{i+1} =  x_i - \frac{f_i(x_i)}{f'_i(x_i)}  \rightarrow x' = x - \frac{a-\frac{1}{x^2}}{\frac{2}{x^3}}
\]

which in turn gives us:


\[

x = x - \left( \frac{ax^3 - x}{2} \right) = \left( \frac{x}{2} \right) \left( 3 - ax^2 \right) \pmod{p^{2k}}

\]

We can slightly modify previous right hand side to get:

\[
x' = x + \frac{x}{2} \left( 1 - ax^2 \right) \pmod{p^{2k}}
\]
And so after we square both sides and multiply through by $a$ we get:


\[
ax'^2 = ax^2 + ax^2 \left( 1 - ax^2 \right) + \left( 1 - ax^2 \right)^2 \pmod{p^{2k}}.
\]

Now we can see that the term on the far right is equivalent to $0 \pmod{p^{2k}}$. If  $1 - ax^2$ is congruent to $0 \pmod{p^{2k}}$, then our work is finished, and we are left with a $1$ on the right hand side of our equality:

\[
ax'^2 \cong 1 \pmod{p^{2k}}.
\]

If this is not the case, then $1 - ax^2 \cong p^k \pmod{p^{2k}}$, and our main equality changes to:

\[
ax'^2 = 1 - p^k + (1-p^k)p^k = 1 - p^k + p^ k -p^{2k} = 1 \pmod{p^{2k}} 
\]

Hence, part $b$ is proved.



\section*{Problem 1}
\subsection*{Part A:}

$L_t(a)$ is clearly linear, since whenever a polynomial is squared, all possible pairs of terms are multiplied together and added to the new sum, and the only time there are not two copies of a these new terms in the new sum is when a term is paired with itself: hence for any $ a \in R/\[t\]$, sending it through the map will only double the degrees of the terms of $a$, before finally subtracting $at$ from this new term, $a^2 \pmod{\zz_2}$. In other words, if $a,b \in R/\[t\]$, then:

\[
L_t(a+b) = \left(a+b\right)^2 - t \left(a + b \right) \pmod{\zz_2} = a^2 + b^2 - ta - tb \pmod{\zz_2}.
\]

And hence $L_t(a)$ is $\zz_2$ linear.

\subsection*{Part B:}

First, suppose that $ a \in \ker(L_x)$, then $a^2 \cong aX \pmod{f}$, and hence this congruency must hold for all $f_1,f_2, \ldots f_i$ irreducible factors of $f$. Since we have 

\[
    a^2 \cong ax \pmod{f_i}
\]
for all $f_i$ irreducible, we can make an observation regarding the form of the $a$ in our congruences. Namely, since $f_i$ is irreducible, it forms a field when it's quotient is taken from the polynomial ring $\zz_2\[x\]$. Hence, either $a$ must have an inverse in $\zz\[x\]/(f)$, which would imply $a = X \pmod{f_i}$, or $a$ must be zero $\pmod{f_i}$, which concludes our proof from left to right.

Now if $a$ is congruent to $0,X \pmod{f_i}$, for some $f_i$, we can use the Chinese Remainder Theorem to create a solution to this system of congruences which will be unique $\pmod{f}$. To do this, we create a sum $\pmod{f}$ which is a chain of terms of the form $e_if_i'x$, where $f_i'$ is the product of all the other irreducible $f_j$ which divide $f$, $e_i$ is the inverse of $f_i' \pmod {f_i}$. This term will create a sort of dimple at $f_i$, being congruent to $x \pmod{f_i}$, and zero everywhere else. Now, if this term is only congruent to $X \pmod{f_i}$ and zero everything else, then it will be congruent to $X \pmod {f}$, and hence will be in the kernel of $L_X$, since $X^2 - X^2 = 0 \pmod{f}$. Since $L_X$ is $\zz_2$ linear, the sum all of the individual dimples that are added together to mimic the behavior of a single polynomial over the smaller congruences will be mapped as if they were sent through the function individually and added together in the image. Since each one on its own is sent to zero, their sum in the image will be zero, which means their sum, if taken before the function, is in the kernel.



\end{document}   
