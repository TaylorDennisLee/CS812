\documentclass[12pt]{article}
\usepackage{amsmath,amsfonts,amsthm}
\usepackage[fleqn]{mathtools}
\usepackage{a4wide}
\usepackage{fancyhdr} % Custom headers and footers
\usepackage{graphicx}
\usepackage{listings}
\usepackage{parskip}

\usepackage[adobe-utopia]{mathdesign}
\usepackage[T1]{fontenc}

\usepackage{titlesec} % www.ctan.org/tex-archive/macros/latex/contrib/titlesec/titlesec.pdf

% Below "\section" can be replaced with "\subsection" and "\subsubsection"
% in order to customize the corresponding headings



\pagestyle{fancyplain}
\fancyhead[L]{HW3: 03/26/2014}
\fancyhead[C]{University of Wisconsin-Madison}
\fancyhead[R]{Professor Eric Bach}
\fancyfoot[L]{Taylor Lee}
\fancyfoot[C]{CS 812}
\fancyfoot[R]{tdlee2@wisc.edu}
\newtheoremstyle{moo}
{10pt}
{10pt}
{}
{}
{\bfseries}
{:}
{\newline }
{}


\theoremstyle{moo}
\newtheorem*{prob}{Problem}
\newtheorem*{sol}{Solution}
\newtheorem*{theorem}{Theorem}

\def\zz{{\mathbb Z}}
\def\lr{{\Leftrightarrow}}
\def\poi{{\pmod{26}}}


\begin{document}
\fontencoding {T1}
\fontfamily {put}
\fontseries {\seriesdefault}
\fontshape {\shapedefault}
%\fontsize {size} {baselineskip} 
\selectfont

\title{\usefont{T1}{put}{b}{n} Arithmetic Algorithms 2014: HW3}
\date{March 26, 2014}         % used by \maketitle
\author{Taylor Lee}      % used by \maketitle                    % used by \maketitle
\maketitle                      % automatic title!




\section*{Problem 1}


\subsection*{Part A:}
I really struggled with this question. This answer came with the asstince of a translation of one of Euler's papers.

\subsection*{Part B:}
Euler could have discovered that $6,700,417$ is prime. He could have found an approximation for this number's square root, and from here see that there would only be ten further numbers of the form $k2^5+1$ that could be the smallest divisor of $6,700,417$.
\subsection*{Part C:}

Since their are 10 different ways to multiply a number by a single-digit number, if each multiple occurs uniformily, then we can estimate the expected number of times we expected to record a randomly occuring multiple before having them all. We have compensated our estimate for the chances that we catch.

\[
1 + 10/9 + 10/8 + 10/7 + 10/6 + 10/5 + 10/4 + 10/3 + 10/2 + 10 = 29.2896825396825
\]

In a sense, however, this is not an accurate prediction of the expected number of multiples we will have to generate before cataloging all ten multiples of $d$. If we were to select $2$ through $9$ before $0$ and $1$, then we would have acheived our goal, because the results of multiplying these two numbers to our divisor $d$ is trivial. Since there are $10$ choose $8$ ways of selecting the first $8$ numbers, and sense only one of these $45$ leaves both $0$ and $1$ until the end, we can make a small correction to our second to last term, as there is a $1/45$ chance we will not have to attempt it. Likewise, with our final term, if we happen to have selected all but one of our two trivial multiples in the prior iterations, then we can skip this final step as well. Since this is also subject to the early-abort scenario of the second to last term, we will also include $44/45$ in this term, to compliment the $4/5$ correction for arriving the end with only a trivial multiple outstanding. Hence, our expected number of iterations prior to recording all single digit multiples of $d$ is:

\[
1 + 10/9 + 10/8 + 10/7 + 10/6 + 10/5 + 10/4 + 10/3 + (10/2)*(44/45) + 10*(44/45)*(8/10) = 27.00079365079365
\]

\section*{Problem 2}

\subsection*{Part A:}

Since we know that we are operating over the Ring of integers $\zz/p^k \zz $, we know that the multiplicative Group of this Ring, $\left(\zz/p^k \zz \right) \ast $, is isomorphic to $\zz/p^{k-1}\zz \times \zz/p-1 /zz $. Hence, for any $a \in \left(\zz/p^k \zz \right) \ast $, we know the order of $a$ must divide either $p^{k-1}$ or $p-1$, if not both.

Now, by squaring both sides of the given congruence, we can quickly see the size of $L(p^k)$.

\[
a^{(p^k-1)/2} \equiv \pm 1 \pmod{p^k} \Rightarrow a^{(p^k-1)} \equiv 1 \pmod{p^k}.
\]

Hence, if $a$ is to be counted in $L(p^k)$, then not only must it be invertible, but it's order must divide $p^k - 1$. Now, we can see that 

\[(p-1)*(p^{k-1} + p^{k-2} + \ldots + p + 1) = p^k - 1 \]

So $a$ can divide $(p-1)$. But since $p^{k-1} \nmid p^{k-1} + p^{k-2} + \ldots + p + 1$, $a$ must divide $(p-1)$ and $(p-1)$ alone. Since there are $p-1$ elements of this type, which satisfy both our congruence condition and our requirements for element order, we can see that $|L(p^k)| = p-1$.

$L$ is not multiplicative. For instance, $L(3) = 2$ and $L(5) = 4$, however $L(15) = 2$.


\subsection*{Part B:}
Consider the prime factorization of $n$.

\[
n = p_1^{e_1}p_2^{e_2} \ldots p_{i_1}^{e_{i-1}}p_i^{e_i}
\]

We know from a structure theorem that 

\[
\left(\zz/n\zz\right)^* \cong 
\]

\end{document}   
